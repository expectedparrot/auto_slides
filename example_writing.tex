There is a long-standing debate about the role advertising plays in markets \citep{bagwell2007economic}.
Critics of advertising contend that advertising can distort consumer preferences, increase costs, concentrate market power, and waste resources.
Those more favorably disposed to advertising argue that advertising helps connect buyers and sellers by conveying market-relevant information.
For example, advertising can inform potential buyers and sellers about each other, or it can allow advertisers to send a costly signal that may reveal information about product quality~\citep{stigler1961economics, nelson1974, milgromroberts1986}.
However, these positive conceptions of advertising may be less relevant in a world where market information is almost free, and abundant product information is available online.
\cite{woodcock2017obsolescence} makes this point in the provocatively and informatively titled law review article ``The Obsolescence of Advertising in the Information Age.''\footnote{The legal stake was that the FTC should view all advertising as persuasive rather than informative.}

In this paper, we consider the role of advertising in the context of a digital marketplace for services. 
For the first time, \fls{} were given the opportunity to buy paid advertising.
The advertisement was a ``badge'' with the text ``Available Now'' that appeared next to a \fl{} in search results.
The \cl{} could also see a notice that the \fl{} paid for this badge. 
Importantly, these paid advertisements did not give \fls{} greater visibility to \cls{}:
advertising did not change search rankings, nor the size of the displays in the search results~\citep{edelman2007internet, athey2011, decarolis}.
In the experimental phase, all \fls{} could advertise---but only randomly treated \cls{} could \emph{see} the advertisements.
We, as the experimenters, know which \fls{} advertised even though untreated \cls{} did not.

The platform hoped that paid advertising would lead to only relatively more available \fls{} advertising.
\CLS{} would, in turn, direct their attention to these advertising \fls{}.
The market failure the platform hoped to overcome is that \cls{} are uncertain about \fls{}' capacities, which causes them to  pursue unavailable \fls{}~\citep{horton2019buyer,fradkin2023search}.
Despite this hope, the platform feared that advertisers would be adversely selected and \cls{} would learn to simply ignore advertisers, which in turn would cause demand for advertising to disappear.

The goal of the initiative was, in economic terms, to try to make search less ``random'' \citep{mortensen1994job} and more ``directed'' \citep{wright2021directed}.
Our primary research question is whether the introduction of advertising improved market efficiency in equilibrium, and if so, why.
Although our context is one particular digital market, the answers to these questions are informative about the role advertising plays in markets more generally, and the economic problem of strategically missing information that we describe is quite general.
